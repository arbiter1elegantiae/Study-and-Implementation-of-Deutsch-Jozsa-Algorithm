\documentclass[12pt,a4paper,openright]{report}
\usepackage[italian]{babel}
\usepackage{newlfont}
\usepackage[utf8]{inputenc}
\usepackage{fancyhdr}
\usepackage{indentfirst}
\usepackage{showkeys}
\usepackage{amssymb}
\usepackage{amsmath}
\usepackage{latexsym}
\usepackage{amsthm}
\usepackage{qcircuit}
\usepackage{braket}
\usepackage{kbordermatrix}
\usepackage{relsize}
\usepackage{listings}
\usepackage{color}
\definecolor{dkgreen}{rgb}{0,0.6,0}
\definecolor{gray}{rgb}{0.5,0.5,0.5}
\definecolor{mauve}{rgb}{0.58,0,0.82}
\renewcommand{\chaptermark}[1]{\markboth{\thechapter.\ #1}{}}
\renewcommand{\sectionmark}[1]{\markright{\thesection \ #1}{}}
\newcommand*{\field}[1]{\mathbb{#1}}
\newcommand*\xor{\mathbin{\oplus}}
\newtheorem{mydef}{Definizione}[chapter]
\newtheorem*{mycor}{Corollario}
\pagestyle{fancy}\addtolength{\headwidth}{30pt}
\rhead[\fancyplain{}{\bfseries\leftmark}]{\fancyplain{}{\bfseries\thepage}}
\cfoot{}
%%%%%%%%%%%%%%%%%%%%%%%%%%%%%%%%%%%%%%%%%
\linespread{1.3}                        %comando per impostare l'interlinea
%%%%%%%%%%%%%%%%%%%%%%%%%%%%%%%%%%%%%%%%%definisce nuovi comandi
%
\textwidth=450pt\oddsidemargin=0pt
\begin{document}
\begin{titlepage}
\begin{center}
{{\Large{\textsc{Alma Mater Studiorum $\cdot$ Universit\`a di
Bologna}}}} \rule[0.1cm]{15.8cm}{0.1mm}
\rule[0.5cm]{15.8cm}{0.6mm}
{\small{\bf SCUOLA DI SCIENZE\\
Corso di Laurea in Informatica }}
\end{center}
\vspace{15mm}
\begin{center}
{\LARGE{\bf TITOLO}}\\
\vspace{3mm}
{\LARGE{\bf DELLA}}\\
\vspace{3mm}
{\LARGE{\bf TESI}}\\
\end{center}
\vspace{40mm}
\par
\noindent
\begin{minipage}[t]{0.47\textwidth}
{\large{\bf Relatore:\\
Chiar.mo Prof.\\
UGO DAL LAGO}}
\end{minipage}
\hfill
\begin{minipage}[t]{0.47\textwidth}\raggedleft
{\large{\bf Presentata da:\\
FEDERICO PECONI}}
\end{minipage}
\vspace{20mm}
\begin{center}
{\large{\bf II Sessione\\%inserire il numero della sessione in cui ci si laurea
a.a. 2016/2017 }}%inserire l'anno accademico a cui si è iscritti
\end{center}
\newpage
\thispagestyle{empty}                   %elimina il numero della pagina
\topmargin=6.5cm                        %imposta il margina superiore a 6.5cm
\raggedleft                             %incolonna la scrittura a destra
\large                                  %aumenta la grandezza del carattere
                                        %   a 14pt
\em                                     %emfatizza (corsivo) il carattere
Questa \`e la \textsc{Dedica}:\\
ognuno pu\`o scrivere quello che vuole, \\
anche nulla \ldots                      %\ldots lascia tre puntini
\newpage                                %va in una pagina nuova
%
%%%%%%%%%%%%%%%%%%%%%%%%%%%%%%%%%%%%%%%%
\clearpage{\pagestyle{empty}\cleardoublepage}%non numera l'ultima pagina sinistra
\end{titlepage}
\tableofcontents
\chapter{Introduzione}
\chapter{Funzioni Booleane Bilanciate}
Come già anticipato nell'Introduzione, le funzioni Booleane bilanciate sono una struttura matematica su cui poggia parte del contenuto di questa tesi.
É risultato perciò utile, ai fini di una trattazione chiara ed esaustiva, impiegare un capitolo per definirne in maniera rigorosa i concetti di base.\\
"\textit{Per fare un tavolo ci vuole il legno}" recita l'inizio di una famosa canzone per bambini, ad indicare la natura celata delle cose che, così nella quotidianità
come nella matematica, spesso necessitano di altre conoscenze per essere comprese appieno; seguendo quindi questa impronta fondazionale, andiamo per prima cosa ad introdurre le funzioni Booleane.

\section{Funzioni Booleane}
Le funzioni Booleane apparvero per la prima volta a metà 19esimo secolo durante la formulazione matematica di problemi logici e prendono
il loro nome da George Boole, matematico britannico considerato fondatore della logica matematica odierna\cite{ref2}.\newpage
Una semplice funzione Booleana può avere la seguente forma \\$f:\{0,1\}^2 \mapsto \{0,1\}$
\begin{align*}
    f(00) = 0 \\
    f(01) = 0 \\
    f(10) = 1 \\
    f(11) = 0
\end{align*}
oppure $f^\prime:\{TRUE,FALSE\}\mapsto{\{TRUE,FALSE\}}$
\begin{align*}
    f^\prime(FALSE) = TRUE \\
    f^\prime(TRUE) = FALSE 
\end{align*}
Notiamo come entrambe abbiano in comune la dimensione del codominio e l'agire su un numero finito di valori appartenenti ad un insieme di 2 elementi.
È tuttavia conveniente operare su di un insieme che possa essere visto sia dal punto di vista qualitativo (vero, falso)
sia da un punto di vista quantitativo, e quindi numerico, che ci permetta così di compiere anche operazioni algebriche oltre che logiche. Prediligeremo allora da qui in avanti, l'inisieme
$\{0,1\}$ come campo vettoriale su cui costruirle.
\par
\begin{mydef}
    Una \textnormal{funzione Booleana a \textit{n} variabili} è una funzione da $\mathcal{B}^n$ a $\mathcal{B}$,
    dove $\mathcal{B} = \{0,1\}$, $n > 0$ e $\mathcal{B}^n$ è l'n-esimo prodotto cartesiano di $\mathcal{B}$ con se stesso.
\end{mydef}
\begin{mycor}
    $\forall$ $n > 0$, ci sono $2^{2^{n}}$ funzioni da $\mathcal{B}^n$ a $\mathcal{B}.$
\end{mycor}
\begin{proof}
    Sia $\mathcal{F}=\{f|f:\mathcal{B}^n\mapsto{\mathcal{B}}\}$,
    ogni $f$ riceve in input n-uple $\vec{x}=(x_1,..,x_n)$ che possono essere viste come sequenze di $n$ bit.
    In $n$ bit possiamo codificare $2^n$ oggetti differenti, quindi $\left\vert{\mathcal{B}^n}\right\vert = 2^n$.
    Per ogni $\vec{x}$, $f(\vec{x}) = 0$  oppure  $f(\vec{x}) = 1$, quindi ogni possibile $f$ sarà funzione caratteristica per $\mathcal{B}^n$, individuandone un sottoinsieme.\\
    Allora $\mathcal{F}$ corrisponde all'insieme delle parti per $\mathcal{B}^n$, quindi  $\left\vert{\mathcal{F}}\right\vert = 2^{\mathcal{B}^n}=2^{2^{n}}$

\end{proof}
Un altro modo più tradizionale per descrivere una funzione Booleana è quello di fornire la sua tabella di verità.
Ad esempio, per la $f$ precedentemente definita, la tabella di verità relativa sarà:


\begin{displaymath}
    \begin{array}{|c|c|c|}\hline
        x_1 & x_2 & f(x_1, x_2) \\\hline 
        0   & 0   &  0  \\ 
        0   & 1   &  0  \\
        1   & 0   &  1  \\
        1   & 1   &  0  \\\hline
    \end{array}
\end{displaymath}

In entrambe le rappresentazioni, tuttavia, le funzioni vengono descritte in maniera implicita mostrando solamente input e output,
senza mai andare a specificare come questo output venga calcolato.\\
Nell'ultimo capitolo, per l'implementazione dell'algoritmo di Deutsch-Jozsa, verranno utilizzate funzioni Booleane \textit{bilanciate}
che necessitano di essere calcolate esplicitamente. La prossima sezione introduce questa categoria di funzioni Booleane e 
propone due metodi effettivi e generali per produrne istanze concrete.

\begin{thebibliography}{20}
    \addcontentsline{toc}{chapter}{Bibliografia}
    \bibitem{ref1} Higham, N. (1998). \emph{Handbook of writing for the mathematical sciences.} Philadelphia: SIAM, Soc. for Industrial and Applied Mathematics.
    \bibitem{ref2} Encyclopediaofmath.org. (2017). \emph{Boolean function - Encyclopedia of Mathematics.} [online] 
    \bibitem{ref3} nono libro
\end{thebibliography}
\end{document}