\documentclass[12pt,a4paper,openright,twoside]{report}
\usepackage[italian]{babel}
\usepackage{newlfont}
\usepackage[utf8]{inputenc}
\usepackage{fancyhdr}
\usepackage{indentfirst}
\usepackage{showkeys}
\usepackage{amssymb}
\usepackage{amsmath}
\usepackage{latexsym}
\usepackage{amsthm}
\usepackage{qcircuit}
\usepackage{braket}
\usepackage{kbordermatrix}
\usepackage{relsize}
\usepackage{listings}
\usepackage{color}
\definecolor{dkgreen}{rgb}{0,0.6,0}
\definecolor{gray}{rgb}{0.5,0.5,0.5}
\definecolor{mauve}{rgb}{0.58,0,0.82}
\renewcommand{\chaptermark}[1]{\markboth{\thechapter.\ #1}{}}
\renewcommand{\sectionmark}[1]{\markright{\thesection \ #1}{}}
\newcommand*{\field}[1]{\mathbb{#1}}
\newcommand*\xor{\mathbin{\oplus}}
\pagestyle{fancy}\addtolength{\headwidth}{20pt}
\rhead[\fancyplain{}{\bfseries\leftmark}]{\fancyplain{}{\bfseries\thepage}}
\cfoot{}
%%%%%%%%%%%%%%%%%%%%%%%%%%%%%%%%%%%%%%%%%
\linespread{1.3}                        %comando per impostare l'interlinea
%%%%%%%%%%%%%%%%%%%%%%%%%%%%%%%%%%%%%%%%%definisce nuovi comandi
%
\textwidth=450pt\oddsidemargin=0pt
\begin{document}
\begin{titlepage}
\begin{center}
{{\Large{\textsc{Alma Mater Studiorum $\cdot$ Universit\`a di
Bologna}}}} \rule[0.1cm]{15.8cm}{0.1mm}
\rule[0.5cm]{15.8cm}{0.6mm}
{\small{\bf SCUOLA DI SCIENZE\\
Corso di Laurea in Informatica }}
\end{center}
\vspace{15mm}
\begin{center}
{\LARGE{\bf TITOLO}}\\
\vspace{3mm}
{\LARGE{\bf DELLA}}\\
\vspace{3mm}
{\LARGE{\bf TESI}}\\
\end{center}
\vspace{40mm}
\par
\noindent
\begin{minipage}[t]{0.47\textwidth}
{\large{\bf Relatore:\\
Chiar.mo Prof.\\
UGO DAL LAGO}}
\end{minipage}
\hfill
\begin{minipage}[t]{0.47\textwidth}\raggedleft
{\large{\bf Presentata da:\\
FEDERICO PECONI}}
\end{minipage}
\vspace{20mm}
\begin{center}
{\large{\bf II Sessione\\%inserire il numero della sessione in cui ci si laurea
a.a. 2016/2017 }}%inserire l'anno accademico a cui si è iscritti
\end{center}
\newpage
\thispagestyle{empty}                   %elimina il numero della pagina
\topmargin=6.5cm                        %imposta il margina superiore a 6.5cm
\raggedleft                             %incolonna la scrittura a destra
\large                                  %aumenta la grandezza del carattere
                                        %   a 14pt
\em                                     %emfatizza (corsivo) il carattere
Questa \`e la \textsc{Dedica}:\\
ognuno pu\`o scrivere quello che vuole, \\
anche nulla \ldots                      %\ldots lascia tre puntini
\newpage                                %va in una pagina nuova
%
%%%%%%%%%%%%%%%%%%%%%%%%%%%%%%%%%%%%%%%%
\clearpage{\pagestyle{empty}\cleardoublepage}%non numera l'ultima pagina sinistra
\end{titlepage}
\tableofcontents
\chapter{Introduzione}
\chapter{Funzioni Booleane Bilanciate}
Come già anticipato nell'Introduzione, le funzioni Booleane bilanciate sono una struttura matematica su cui poggia parte del contenuto di questa tesi.
É risultato perciò utile, ai fini di una trattazione chiara ed esaustiva, impiegare un capitolo per definirne in maniera rigorosa i concetti di base.\\
"\textit{Per fare un tavolo ci vuole il legno}" recita l'inizio di una famosa canzone per bambini, ad indicare la natura celata delle cose che, così nella quotidianità
come nella matematica, necessitano di altre conoscenze per essere comprese appieno; seguendo quindi questa impronta fondazionale, andiamo per prima cosa ad introdurre le funzioni Booleane.

\section{Funzioni Booleane}
Una molto semplice funzione Booleana può essere rappresentata dalla seguente \\$f:\{0,1\}^2 \mapsto \{0,1\}$
\begin{align*}
    f(00) = 0 \\
    f(01) = 0 \\
    f(10) = 1 \\
    f(11) = 0
\end{align*}
\par
Una funzione Booleana è definita formalmente nel seguente modo:\\
Sia $ \field{N} $ l'insieme di tutti gli interi positivi e $ \field{N\ped{0}} = \field{N} \cup\{0\} $.
Per un arbitrario $n\in{\field{N\ped{0}}}$, denotiamo $\field{F\ped{2}}$ un campo vettoriale a due elementi
e con $\mathcal{V\ped{n}} = \field{F\ped{2}}^n$ lo spazio vettoriale
di tutte le stringhe di lunghezza $n$ con elementi appartenenti a $\field{F\ped{2}}$. Le operazioni di campo
in $\field{F\ped{2}}$ sono denotate da $\xor$ per la somma (modulo 2) e $\cdot$ per la moltiplicazione (\small{”}),
 lo stesso simbolo $\xor$ denota l'addizione (\small{”}) anche tra vettori in $\field{F\ped{2}}^n$. \\ 
Una \textit{funzione Booleana $f$} è allora un'applicazione lineare $f: \mathcal{V\ped{n}} \mapsto{\field{F\ped{2}}}$ per qualche $n \in \field{N\ped{0}}$.

\end{document}